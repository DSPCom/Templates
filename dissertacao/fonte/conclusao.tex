\chapter*[Conclusão]{Conclusão}
\addcontentsline{toc}{chapter}{Conclusão}

Este trabalho de doutorado é o resultado do estudo e proposição de algoritmos de ITL para abordar os problemas não-supervisionados de separação de sinais em corpos finitos e inversão de sistemas de Wiener. Dadas as características dos dois problemas mencionados, desenvolvemos contribuições que se baseiam não somente no paradigma tradicional de ITL --- algoritmos de ajuste por gradiente e medidas de informação de Rényi ---, mas também incorporam as estratégias imuno-inspiradas de busca, a melhoria de heurísticas de busca específicas, e o emprego de estimadores de TI baseados nas definições de Shannon.

Para levar a termo esta proposta de trabalho, a tese acaba tocando, ao longo de seu conteúdo, em aspectos de diversas disciplinas como Álgebra, Estatística, Teoria da Informação, Processamento de Sinais, Filtragem Adaptativa, Inteligência Computacional, e Complexidade de Algoritmos.

Recapitulando o conteúdo da tese, no \autoref{cap:ITL} foram apresentados os aspectos históricos e definições essenciais da Teoria da Informação, e o vínculo posterior dessa disciplina primordial com o desenvolvimento da nova área de Aprendizado de Máquina Baseado na Teoria da Informação, iniciada no contexto da extensão de critérios de algoritmos adaptativos baseados em ESO. Por fim, o capítulo se encerra propondo a ampliação do escopo desta nova área de pesquisa.

Os dois capítulos subsequentes, \ref{cap:SSCF} e \ref{cap:WH}, são dedicados à apresentação formal dos problemas que são abordados na tese, a separação cega de sinais em corpos finitos e a inversão cega de sistemas de Wiener, respectivamente. Além de descrevê-los matematicamente, apresentou-se uma revisão bibliográfica acerca das principais abordagens para cada problema, acompanhada do levantamento de fatores que geram as oportunidades de contribuições pautadas na tese.

O \autoref{cap:SIA} encerrou a parte referente aos Fundamentos apresentando uma introdução à computação imuno-inspirada, em especial aos Sistemas Imunológicos Artificiais e a seus representantes CLONALG e cob-aiNet, algoritmos que são empregados em algumas das contribuições analisadas na parte seguinte.

O \autoref{cap:SSCF-cont}, primeiro da parte de Contribuições, apresentou os trabalhos desenvolvidos para ICA e BSS em corpos finitos, que trazem soluções baseadas no (\emph{i}) algoritmo CLONALG com o critério de mínima IM, no (\emph{ii}) algoritmo cob-aiNet[C] com o critério de mínima entropia, e no (\emph{iii}) novo algoritmo MEXICO-m. As simulações para os três algoritmos indicaram que os métodos possuem desempenho competitivo frente às técnicas existentes, tornando mais concreta a possibilidade de aplicação em cenários com números maiores de fontes. Ademais, foi proposta a extensão do problema de BSS original para o modelo de misturas convolutivas, incluindo aí a análise da questão estrutural de inversibilidade e de um algoritmo iterativo para extração dos sinais. Portanto, podemos reafirmar que os trabalhos que se desenvolveram possuem o objetivo comum de trazer maior robustez e escalabilidade a ICA em corpos finitos.

Por fim, o \autoref{cap:WH-cont} apresenta a proposta de uma nova metodologia para a inversão cega de sistemas de Wiener, baseada no algoritmo CLONALG e estimadores de IM como função objetivo do problema de otimização dos parâmetros. Além da estruturação deste \emph{framework}, que possui flexibilidade quanto à modelagem do sistema de Hammerstein no que diz respeito à não-linearidade e modelo de filtro (FIR ou IIR), realizamos um estudo qualitativo da relação entre o critério MMI e o critério MMSE para a inversão do canal linear, que forneceu indícios da viabilidade do algoritmo para o cenário não-linear. Uma série de simulações foi realizada, incluindo a comparação com as técnicas estado-da-arte, as quais apontaram que o novo método apresenta desempenho equivalente ou até superior nos cenários avaliados.

\section*{Perspectivas Futuras}

Dados os resultados, há uma série de trabalhos futuros possíveis de se realizar, e listamos a seguir alguns deles.

Com respeito ao problema de separação de sinais em corpos de Galois:
\begin{itemize}
	\item É importante analisar em detalhes as conexões práticas e conceituais entre o processamento de sinais em corpos finitos e a teoria de codificação.
	\item Uma perspectiva de aplicação real consiste em empregar ICA em corpos finitos como uma ferramenta para realizar análise de fatores de dados discretos. Esta possibilidade pode ter impacto significativo na área de mineração de dados em pelo menos dois domínios de grande impacto: grandes bases de dados armazenadas em formato digital e bases de dados genômicos. No primeiro caso, uma escolha natural seria $GF(2)$, mas, no segundo caso, a idéia consistiria em determinar uma matriz capaz de explicar uma massa de dados como uma combinação de fatores independentes em $GF(3)$ ou $GF(2)$, sendo a ordem do corpo definida em concordância com a cardinalidade do conjunto factível de pares de bases nitrogenadas do DNA. Esta possibilidade parece também ser relevante à vista dos recentes esforços~\cite{Faria2010} para se associar a Teoria da Informação à Genética.
	\item No caso específico da proposta de ICA com a cob-aiNet[C], é interessante futuramente estender o algoritmo para sinais em corpos não necessariamente primos, aprofundar o estudo da metodologia com diferentes métricas de dissimilaridade, em especial aquelas que operam no espaço fenotípico, e simplificar o processo de determinação dos parâmetros de operação do algoritmo.
	\item É importante que se analise de maneira mais completa a relação entre o número de amostras disponíveis e o desempenho atingível por métodos operando sobre corpos finitos. Isso também motiva a busca por estimadores mais eficientes e a aplicação sistemática de abordagens já propostas, inclusive o método de estimação de entropia desenvolvido, em caráter de cooperação, durante o trabalho de doutorado~\cite{Montalvao2012}.
	\item Em consonância com a estratégia de explorar novas formulações de misturas de sinais em corpos de Galois, como realizado com o caso de misturas convolutivas, é interessante avaliar a possibilidade de lidar com modelos de mistura não-lineares e inversíveis. Uma perspectiva interessante nesse sentido parece ser buscar diretamente permutações das possíveis entradas do sistema misturador com a ajuda de algoritmos de otimização combinatória.
	\item Finalmente, há a perspectiva de estudar os problemas de separação de sinais e ICA dentro de formulações com outras estruturas algébricas, como as Álgebras de Lie~\cite{SanMartin2010} e as extensões de corpos~\cite{Fraleigh2002}.
\end{itemize}

Com respeito ao problema de inversão de sistemas de Wiener:
\begin{itemize}
	\item Estender a análise comparativa entre as metodologias existentes, para que contemple um maior número de cenários de canais, sinais de origem e não-linearidades.
	\item Estudar a aplicação de estimadores de IM de custo computacional menor no \emph{framework} proposto nesta tese, a fim de aumentar o número de sinais atrasados considerados no cálculo da função objetivo. Neste sentido, o trabalho desenvolvido em \cite{Montalvao2013} --- que propõe um estimador de entropia \emph{diferencial} baseado em contagem de coincidências --- apresenta-se como uma possibilidade promissora, pois é mais simples e menos custoso que  métodos de estimação baseados em Parzen, mesmo em dimensões altas, podendo fornecer uma qualidade de estimação comparável a estimadores mais sofisticados.
	\item É interessante realizar uma análise abrangente do risco de a busca por gradiente levar a soluções localmente ótimas para o problema, análise esta que seria importante aliar a uma avaliação do papel de meta-heurísticas frente a tal limitação.
	\item É possível estudar novas medidas de independência no papel de função objetivo do problema, como, por exemplo, a correntropia~\cite{Santamaria2006}. Para o caso de sinais advindos de sistemas determinísticos, também seria relevante considerar métricas desenvolvidas no estudo de dinâmica não-linear como mapas de retorno~\cite{Marwan2007}.
	\item A conexão entre o problema de Wiener-Hammerstein e o problema de BSS para misturas PNL precisa ser desenvolvida no sentido de estabelecer algoritmos comuns aos dois métodos e para que se demonstre, de fato, a capacidade de inversão cega do sistema de Wiener por meio da recuperação de independência, o que hoje é apenas conjecturado.
\end{itemize}

