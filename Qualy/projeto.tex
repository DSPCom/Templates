%% abtex2-modelo-projeto-pesquisa.tex, v-1.9.6 laurocesar
%% Copyright 2012-2016 by abnTeX2 group at http://www.abntex.net.br/ 
%%
%% This work may be distributed and/or modified under the
%% conditions of the LaTeX Project Public License, either version 1.3
%% of this license or (at your option) any later version.
%% The latest version of this license is in
%%   http://www.latex-project.org/lppl.txt
%% and version 1.3 or later is part of all distributions of LaTeX
%% version 2005/12/01 or later.
%%
%% This work has the LPPL maintenance status `maintained'.
%% 
%% The Current Maintainer of this work is the abnTeX2 team, led
%% by Lauro César Araujo. Further information are available on 
%% http://www.abntex.net.br/
%%
%% This work consists of the files abntex2-modelo-projeto-pesquisa.tex
%% and abntex2-modelo-references.bib
%%

% ------------------------------------------------------------------------
% ------------------------------------------------------------------------
% abnTeX2: Modelo de Projeto de pesquisa em conformidade com 
% ABNT NBR 15287:2011 Informação e documentação - Projeto de pesquisa -
% Apresentação 
% ------------------------------------------------------------------------ 
% ------------------------------------------------------------------------

\documentclass[
	% -- opções da classe memoir --
	article,
	12pt,				% tamanho da fonte
	openright,			% capítulos começam em pág ímpar (insere página vazia caso preciso)
	oneside,			% para impressão em recto e verso. Oposto a oneside
	a4paper,			% tamanho do papel. 
	% -- opções da classe abntex2 --
	%chapter=TITLE,		% títulos de capítulos convertidos em letras maiúsculas
	%section=TITLE,		% títulos de seções convertidos em letras maiúsculas
	%subsection=TITLE,	% títulos de subseções convertidos em letras maiúsculas
	%subsubsection=TITLE,% títulos de subsubseções convertidos em letras maiúsculas
	% -- opções do pacote babel --
	english,			% idioma adicional para hifenização
	french,				% idioma adicional para hifenização
	spanish,			% idioma adicional para hifenização
	brazil,				% o último idioma é o principal do documento
	]{abntex2}
	
% ---
% PACOTES
% ---

% ---
% Pacotes fundamentais 
% ---
\usepackage{lmodern}			% Usa a fonte Latin Modern
\usepackage[T1]{fontenc}		% Selecao de codigos de fonte.
\usepackage[utf8]{inputenc}		% Codificacao do documento (conversão automática dos acentos)
\usepackage{indentfirst}		% Indenta o primeiro parágrafo de cada seção.
\usepackage{color}				% Controle das cores
\usepackage{graphicx}			% Inclusão de gráficos
\usepackage{microtype} 			% para melhorias de justificação
\usepackage{array}
\usepackage{tabularx}
\usepackage{nomencl}
\usepackage{amsmath}
\usepackage{amssymb}
\usepackage{amsfonts}
\usepackage{bbm}
\usepackage[chapter]{algorithm}
\usepackage{algorithmic}
\usepackage{multirow}
\usepackage{rotating}
\usepackage{pdfpages}
\usepackage{bbold}
\usepackage{subfig}
% ---

% ---
% Pacotes de citações
% ---
\usepackage[brazilian,hyperpageref]{backref}	 % Paginas com as citações na bibl
\usepackage[alf]{abntex2cite}	% Citações padrão ABNT

% --- 
% CONFIGURAÇÕES DE PACOTES
% --- 

\newcolumntype{L}[1]{>{\raggedright\let\newline\\\arraybackslash\hspace{0pt}}p{#1}}
\newcolumntype{C}[1]{>{\centering\let\newline\\\arraybackslash\hspace{0pt}}p{#1}}
\newcolumntype{R}[1]{>{\raggedleft\let\newline\\\arraybackslash\hspace{0pt}}p{#1}}

\newcommand{\cvitem}[2]{
\raggedleft
\begin{tabular}{ R{3cm} L{12cm} }
\textbf{#1} & #2 \hfil
\end{tabular}}

%%

\renewcommand{\folhaderostocontent}{

\begin{minipage}{.2\textwidth}
\noindent \includegraphics[width=1.5cm]{unicamp.eps}
\end{minipage}
\hspace*{\fill}
\begin{minipage}{.7\textwidth}
{\large \imprimirinstituicao}
\end{minipage}

    \vspace*{1cm}

\begin{center}
%{\ABNTEXchapterfont\large\imprimirautor}

\vspace*{\fill}\vspace*{\fill}

\ABNTEXchapterfont\bfseries\Large\imprimirtitulo
\end{center}

\vspace*{\fill}

\hspace{.4\textwidth}
\begin{minipage}{.5\textwidth}
\SingleSpacing
\imprimirpreambulo
\end{minipage}
\vspace*{\fill}

\vspace*{1cm}

\noindent
\textbf{Aluno:} \imprimirautor \\
\textbf{Orientador:} \imprimirorientador \\
\textbf{Data de início:}  \\
\textbf{Vigência:} 

\vspace*{\fill}

\begin{center}

{\large\imprimirlocal}

\par

{\large\imprimirdata}
\vspace*{1cm}
\end{center}
}

% ---
% Informações de dados para CAPA e FOLHA DE ROSTO
% ---
\titulo{Template}
\autor{Kenji Nose Filho}
\local{Campinas}
\data{Abril de 2017}
\orientador{Prof. Dr. João Marcos Travassos Romano}
\instituicao{%
    UNIVERSIDADE ESTADUAL DE CAMPINAS
    \par
    Faculdade de Engenharia El\'{e}trica e de Computa\c{c}\~{a}o	
    }
\preambulo{Template de projeto de pesquisa}
% ---

% ---
% Configurações de aparência do PDF final

% alterando o aspecto da cor azul
\definecolor{blue}{RGB}{41,5,195}

% informações do PDF
\makeatletter
\hypersetup{
     	%pagebackref=true,
		pdftitle={\@title}, 
		pdfauthor={\@author},
    	pdfsubject={\imprimirpreambulo},
	    pdfcreator={LaTeX with abnTeX2},
		pdfkeywords={abnt}{latex}{abntex}{abntex2}{Memorial}, 
		colorlinks=true,       		% false: boxed links; true: colored links
    	linkcolor=blue,          	% color of internal links
    	citecolor=blue,        		% color of links to bibliography
    	filecolor=magenta,      		% color of file links
		urlcolor=blue,
		bookmarksdepth=4
}
\makeatother
% --- 

% --- 
% Espaçamentos entre linhas e parágrafos 
% --- 

% O tamanho do parágrafo é dado por:
\setlength{\parindent}{1.3cm}

% Controle do espaçamento entre um parágrafo e outro:
\setlength{\parskip}{0.2cm}  % tente também \onelineskip

% ---
% compila o indice
% ---
\makeindex
% ---

% ----
% Início do documento
% ----
\begin{document}

% Seleciona o idioma do documento (conforme pacotes do babel)
%\selectlanguage{english}
\selectlanguage{brazil}

% Retira espaço extra obsoleto entre as frases.
\frenchspacing 

% ----------------------------------------------------------
% ELEMENTOS PRÉ-TEXTUAIS
% ----------------------------------------------------------
% \pretextual

% ---
% Folha de rosto
% ---
\imprimirfolhaderosto
% ---

% ---
% inserir o sumario
% ---
\pdfbookmark[0]{\contentsname}{toc}
\tableofcontents*
% ---

% ----------------------------------------------------------
% ELEMENTOS TEXTUAIS
% ----------------------------------------------------------
\textual
\setcounter{page}{1}
% ----------------------------------------------------------
% Introdução
% ----------------------------------------------------------
\section{Resumo}
  
\section{Introdução e Justificativa}

\section{Objetivos e plano de trabalho}

\section{Cronograma de atividades}

\begin{enumerate}

\item{} 
	\begin{enumerate}
	\item{} 
	\item{} 
	\item{} 
	\end{enumerate}
\item{} 
	\begin{enumerate}
	\item{} 
	\item{} 
	\item{} 
	\end{enumerate}
\item{} 
	\begin{enumerate}
	\item{} 
	\item{} 
	\item{} 
	\end{enumerate}
\end{enumerate}

\begin{table}[!htpb]
\centering
\begin{small} 
\setlength{\tabcolsep}{3pt}

\begin{tabular}{|c|c|c|c|c|c|c|c|c|c|c|c|c|c|c|c|c|c|c|c|c|c|c|c|c|}\hline
& \multicolumn{24}{c|}{MÊS}\\ \cline{2-25}
\raisebox{1.5ex}{ETAPA} & 01 & 02 & 03 & 04 & 05 & 06 & 07 & 08 & 09 & 10 & 11 & 12 & 13 & 14 & 15 & 16 & 17 & 18 & 19 & 20 & 21 & 22 & 23 & 24 \\ \hline

1.1 & x & x & x & x & x & x &   &   &   &   &   &   &   &   &   &   &   &   &   &   &   &   &   &   \\ \hline
1.2 &   &   & x & x & x & x & x & x &   &   &   &   &   &   &   &   &   &   &   &   &   &   &   &   \\ \hline
1.3 &   &   &   &   & x & x & x & x & x & x &   &   &   &   &   &   &   &   &   &   &   &   &   &   \\ \hline

2.1 &   &   &   &   &   &   &   & x & x & x & x & x & x &   &   &   &   &   &   &   &   &   &   &   \\ \hline
2.2 &   &   &   &   &   &   &   &   &   & x & x & x & x & x & x &   &   &   &   &   &   &   &   &   \\ \hline
2.3 &   &   &   &   &   &   &   &   &   &   &   & x & x & x & x & x & x &   &   &   &   &   &   &   \\ \hline

3.1 &   &   &   &   &   &   &   &   &   &   &   &   &   &   & x & x & x & x & x & x &   &   &   &   \\ \hline
3.2 &   &   &   &   &   &   &   &   &   &   &   &   &   &   &   &   & x & x & x & x & x & x &   &   \\ \hline
3.3 &   &   &   &   &   &   &   &   &   &   &   &   &   &   &   &   &   &   & x & x & x & x & x & x \\ \hline

\end{tabular} 
\end{small}
\caption{Cronograma das atividades previstas}
\label{t_cronograma}
\end{table}

\section{Conclusões}

\postextual

% ---- Refer\^{e}ncias bibliogr\'{a}ficas ----
\bibliography{myref}

\end{document}


